\section{External specification}

The program is to be run in UNIX by typing {\tt pphs} followed by the
filename containing the Haskell code requiring to be typeset.  This will
produce the \LaTeX\ code to stdout.  If there is some error,
a suitable error message is to be printed to stderr.  The user may, if
desired, direct the output to another file by typing {\tt pphs infilename > outfilename}.
In this case, any error messages must still go to the screen and not the file.

The input filename may be given in its entirety or the {\tt .hs} extension may be omitted.
In the case where there are two files with the same name, except that one has the
{\tt .hs} extension, to run the program on the file with the extension to its name
the complete filename will be typed.

The output will consist of the \LaTeX\ code to produce the typeset Haskell
code.  As this is to be made easily insertable into another \LaTeX\ document, the output
will not contain any header information such as declarations or definitions.  These,
however, can
be contained in a style file which the user will include in their main document.

Keywords and identifiers are to be distinguished in the result as typeset.
The default for keywords is to be boldface and for identifiers italics.
Numbers not forming part of an identifier are to be in roman by default
while math is to be used where appropriate.

Haskell uses ASCII characters and combinations of ASCII characters 
to substitute for mathematical characters not present on the
keyboard.  Where this happens, the program is to replace the ASCII character(s)
with the corresponding mathematical character using the special \LaTeX\ commands
to generate them.  The single characters are:
\begin{quote}
\begin{tabular}[t]{@{}cc@{}}
Haskell & Math\\
{\tt *} & $\times$
\end{tabular}
\end{quote}
The double characters are:
\begin{quote}
\begin{tabular}[t]{@{}cc@{}}
Haskell & Math\\
{\tt ++} & {\hbox{$+\mkern-7.5mu+$}}\\
{\tt :+} & {:}{+}\\
{\tt <=} & $\leq$\\
{\tt >=} & $\geq$\\
{\tt <-} & $\leftarrow$\\
{\tt ->} & $\rightarrow$\\
{\tt =>} & $\Rightarrow$
\end{tabular}
\end{quote}

The \LaTeX\ system uses special characters to aid with the typesetting.  
They are:
\begin{quote}
\(\#\ \$\ \%\ \&\ \char'176\ \_\ \char'136\ \hbox{$\setminus$}\ \hbox{$\cal \char'146\ \char'147$}\)
\end{quote}
These characters may
appear in the input, so the program must generate the correct \LaTeX\ code to
print them and
avoid having them mess up the typesetting process.

As the output when typeset must have the same layout as the input, the program
must get the linebreaks and indentation right.  As \LaTeX\ is primarily designed for normal
text, it would ignore the linebreaks and indentation in the Haskell file.  Therefore 
the program must insert them using the correct typesetting commands.  In the case of
linebreaks it must recognise where these occur, but for indentation it must also work out
how much space needs to be inserted.

There are two types of indentation in Haskell programs: left-hand and internal.
For the former, the program must work out what the start of the line is aligned
under in the input file.  It then has to calculate how much space is required
to get the line of text to line up with this in the output once typeset.  
Take, for instance, the following Haskell example input:
\begin{quote}
\begin{verbatim}
foobar a b = c
           where
               c = (a, b)
\end{verbatim}
\end{quote}
Notice that the {\tt w} of {\tt where} on the second line lines up
under the {\tt =} on
the first line.  Similarly, the {\tt c} on the third line is aligned under the
final letter of {\tt where} on the second line.  The result as typeset must
get the indentation correct like this:
\begin{quote}
\begin{tabbing}
foobar a b = c\\
\newbox\foo
{\setbox\foo\hbox{foobar a b }\hskip\wd\foo}where\\
{\setbox\foo\hbox{foobar a b wher}\hskip\wd\foo}c = (a, b)
\end{tabbing}
\end{quote}

For internal indentation, the program must first recognise where it has 
occurred.  It must then insert the correct amount of space to get alignment
in the output.  As \LaTeX\ uses variable-width characters, extra space
may be needed in lines preceding a line within an internal alignment section.  
This is necessary if a lower line which 
aligns in the input file is longer up to the alignment point,
due to the variable-width characters, than its predecessors 
once it has been properly typeset.  For example:
\begin{quote}
\begin{verbatim}
lilli  :: a
wmwm   :: b
\end{verbatim}
\end{quote}
becomes
\begin{quote}
\begin{tabular}[t]{@{}l@{\ }c@{\ }l}
lilli & :: & a\\
wmwm  & :: & b\\
\end{tabular}
\end{quote}
Notice how {\tt lilli} is longer than {\tt wmwm} in the input file style
using fixed-width font but shorter when using the variable-width font
of the typeset output.
