\section{Problem definition}

The problem is that a system is needed to typeset Haskell programs
to be inserted into documents.  This would be useful in, for
instance, preparing papers for publication that are to include
Haskell programs. 

Haskell is a fairly new functional programming language and does not
as yet have a full range of tools available to use with the language.

Many people use the \LaTeX\ system for typesetting.  This uses
embedded typesetting commands in the input to arrange the typesetting.
The result as typeset has variable-width characters with a choice of
font styles and sizes available.  The page-size, margins and layout
are also controllable by the user.

Haskell programs are generally written on editors that produce ASCII
text.  This has fixed-width characters and one plain font.

In Haskell, the language avoids using
line terminators by having indentation to indicate the contextual meaning of
each line. It is thus crucial that this indentation is retained
when the text is put into \LaTeX.  However as the \LaTeX\ system uses
variable width characters, the indentation
level is dependent on the characters under which the text is aligned.
The tabs and spaces that went to make
up the indentation in the original file have to be replaced with a 
suitable amount of space to make the text line up with the position with which it
is aligned in the original file. 

It is also desirable to have
formatting improvements such as highlighting keywords and identifiers as well as
inserting proper mathematical characters in place of the Haskell-ASCII simulations.

Currently the only way of doing this is by labouriously inserting formatting
commands into the text by hand.  The alternative is to print out the programs
verbatim with plain ASCII-style fixed-width font.
