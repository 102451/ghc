\documentclass{article}

\usepackage{color}
\usepackage{fullpage}
\usepackage{hyperref}

\newcommand{\Red}[1]{{\color{red} #1}}

\title{The Backpack Manual}

\begin{document}

\maketitle

\paragraph{What is this?}  This is an in-depth technical specification of
all of the new components associated with Backpack, a new module system
for Haskell.  This is \emph{not} a tutorial, and it assumes you are
familiar with the basic motivation and structure of Backpack.

\paragraph{How to read this manual}  This manual is split into three
sections, in dependency order.  The first section describes the new
features added to GHC, e.g., new compilation flags and input formats.
In principle, a user could take advantage of Backpack using just these
features, without using Cabal or cabal-install; thus, we describe it
first.  The next section describes the new features added to the library
Cabal, and the last section describes how cabal-install drives the
entire process.  A downside of this approach is that we start off by
describing low-level GHC features which are quite dissimilar from the
high-level Backpack interface, but we're not really trying to explain
Backpack to complete new users.  \Red{Red indicates features which are
not implemented yet.}

\section{GHC}

\subsection{Signatures}

An \texttt{hsig} file represents a (type) signature for a Haskell
module, containing type signatures, data declarations, type classes,
type class instances, but not value definitions.\footnote{Signatures are
the backbone of the Backpack module system.  A signature can be used to
type-check client code which uses a module (without the module
implementation), or to verify that an implementation upholds some
signature (without a client implementation.)} The syntax of an
\texttt{hsig} file is similar to an \texttt{hs-boot} file.  Here is an
example of a module signature representing an abstract map type:

\begin{verbatim}
module Map where
type role Map nominal representational
data Map k v
instance Functor (Map k)
empty :: Map k a
\end{verbatim}

For entities that can be explicitly exported and imported, the
export list of a module signature behaves in the same way as
the export list for a normal module (e.g., if no list is provided,
only entities defined in the signature are made available.)

\begin{color}{red}
However, type class instances and type family instances operate
differently: an instance is \emph{only} exported if it is directly
defined in the signature.  This is in contrast to the module behavior,
where an instance is \emph{implicitly} brought into scope if it is
imported in any way (even with an empty import list.)

Even if an instance is ``hidden'' (i.e., not exported by a signature
but in the implementation), we still take it into account when calculating
conflicting instances (e.g., the soundness checks for type families).  Thus,
some compilation errors may only occur when linking an implementation
and user, even if they compiled individually fine against the signature
in question.
\end{color}

An \texttt{hsig} file can either be type-checked or compiled against some
\emph{backing implementation}, an \texttt{hs} module which provides all
of the declarations that a signature advertises.

\paragraph{Typechecking} A signature file can be type-checked in the same
way as an ordinary Haskell file:

\begin{verbatim}
$ ghc -c Map.hsig -fno-code -fwrite-interface
\end{verbatim}

This procedure generates an interface file, which can be used to type
check other modules which depend on the signature, even if no backing
implementation is available.  By default, this generated interface file
is given \emph{fresh} original names for everything in the signature.
For example, if \texttt{data T} is defined in two signature files
\texttt{A.hsig} and \texttt{B.hsig}, they would not be considered
type-equal, and could not be used interconvertibly, even if they
had the same structure.

\begin{color}{red}
To explicitly specify what original name should be assigned (e.g.,
to make the previous example type-equal) the \texttt{-shape-of} flag
can be used:

\begin{verbatim}
$ ghc -c Map.hsig -shape-of "Map is containers_KEY:Data.Map.Map" \
    -fno-code -fwrite-interface
\end{verbatim}

\texttt{-shape-of} is comma separated list of \texttt{name is origname}
entries, where \texttt{name} is an unqualified name and
\texttt{origname} is an original name, of the form
\texttt{package\_KEY:Module.name}, where \texttt{package\_KEY} is a
package key identifying the origin of the identifier (or a fake
identifier for a symbol whose provenance is not known).  Each instance
of \texttt{origname} in the signature is instead assigned the original
name \texttt{origname}, instead of the default original name.

(ToDo: This interface will work pretty poorly with \texttt{--make})
\end{color}

\paragraph{Compiling} We can specify a backing implementation for
a signature and compile the signature against it using the
\texttt{-sig-of} flag:

\begin{verbatim}
$ ghc -c Map.hsig -sig-of "package_KEY:Module"
\end{verbatim}

The \texttt{-sig-of} flag takes as an argument a module, specified
as a package key, a colon, and then the module name.  \Red{This module
must be a proper, \texttt{exposed-module}, and not a reexport or
signature.}

Compilation of a signature entails two things.  First, a consistency
check is performed between the signature and the backing
implementation, ensuring that the implementation accurately implements
all of the types in the signature.  For every declaration in the
signature, there must be an equivalent one in the backing implementation
with an identical type (this check is quite similar to the one used
for \texttt{hs-boot}).  Second, an interface file is generated
which reexports the set of identifiers from the backing
implementation that were specified in the signature. A file which
imports the signature will use this interface file.\footnote{This
interface file is similar to a module which reexports identifiers
from another module, except that we also record the backing implementation
for the purpose of handling imports, described in the next section.}

\begin{color}{red}
ToDo: In what cases is a type class instance/type family instance reexported?
Currently, type classes from the backing implementation leak through.
We also need to fix \#9422.
\end{color}

\subsection{Extended format in the installed package database}\label{sec:pkgdb}

After a set of Haskell modules has been compiled, they can be registered
as a package in the \emph{installed package database} using
\texttt{ghc-pkg}.  An entry in the installed package database specifies
what modules and signatures from the package itself are available for
import. It can also re-export modules and signatures from other
packages.\footnote{Signature reexports are essential for creating
signature packages in a modular way; module reexports are very useful
for backwards-compatibility packages and also taking an package that has
been instantiated multiple ways and giving its modules unique names.}

There are three fields of an entry in the installed package database of note.

\paragraph{exposed-modules} A comma-separated list of
module names which this package makes available for import, possibly with two extra, optional pieces of information
about the module in question: what the \emph{original module/signature}
is (\texttt{from MODULE})\footnote{Knowing the original module/signature
makes it possible for GHC to directly load the interface file, without
having to follow multiple hops in the package database.}, and what the
\emph{backing implementation} is (\texttt{is MODULE})\footnote{Knowing
the backing implementation makes it possible to tell if an import is
unambiguous without having to load the interface file first.}.

\begin{verbatim}
exposed-modules:
    A,                              # original module
    B from ipid:B,                  # reexported module
    C is ipid:CImpl,                # exposed signature
    D from ipid:D is ipid:DImpl,    # reexported signature
    D from ipid:D2 is ipid:DImpl    # duplicates can be OK
\end{verbatim}

If no reexports or signatures are used, the commas can be omitted
(making this syntax backwards compatible with the original syntax.)\footnote{Actually,
the parser is a bit more lenient than that and can figure out commas when it's
omitted. But it's better to just put commas in.}

\paragraph{instantiated-with} A map from hole name to the \emph{original
module} which instantiated the hole (i.e., what \texttt{-sig-of}
parameters were used during compilation.)

\paragraph{key} The \emph{package key} of a
package, an opaque identifier identifying a package
which serves as the basis for type identity and linker
symbols.\footnote{Informally, you might think of a package as a package
name and its version, e.g., \texttt{containers-0.9}; however, sometimes,
it is necessary to distinguish between installed instances of a package
with the same name and version which were compiled with different
dependencies.} When files are compiled as part of a package, the
package key must be specified using the \texttt{-this-package-key}
flag.\footnote{The package key is different from an
\emph{installed package ID}, which is a more fine-grained identifier for
a package.  Identical installed package IDs imply identical package
keys, but not vice versa.  However, within a single run of GHC, we
enforce that package keys and (non-shadowed) installed package IDs are
in one-to-one correspondence.}

The package key is programatically generated by Cabal\footnote{In
practice, a package key looks something like
\texttt{conta\_GtvvBIboSRuDmyUQfSZoAx}.  In this document, we'll use
\texttt{containers\_KEY} as a convenient shorthand to refer to some
package key for the \texttt{containers} package.}. While GHC doesn't
specify what the format of the package key is,  Cabal's must choose distinct package keys if
any of the following fields in the installed package database are
distinct:

\begin{itemize}
\item \texttt{name} (e.g., \texttt{containers})
\item \texttt{version} (e.g., \texttt{0.8})
\item \texttt{depends} (with respect to package keys)
\item \texttt{instantiated-with} (with respect to package keys and module names)
\end{itemize}

\subsection{Module thinning and renaming}

The command line flag \texttt{-package pkgname} causes all
exposed modules of \texttt{pkgname} (from the installed package database) to become visible under their
original names for imports.
The \texttt{-package} flag and its variants (\texttt{-package-id} and
\texttt{-package-key}) support ``thinning and renaming''
annotations, which allows a user to selectively expose only certain
modules from a package, possibly under different names.\footnote{This
feature has utility both with and without Backpack.  The ability to
rename modules makes it easier to deal with third-party packages which
export conflicting module names; under Backpack, this situation becomes
especially common when an indefinite package is instantiated multiple
time with different dependencies.}

Thinning and renaming can be applied
using the extended syntax \verb|-package "pkgname (rns)"|, where \texttt{rns} is a comma separated list of
module renamings \texttt{OldName as NewName}.  Bare module names are
also accepted, where \texttt{Name} is shorthand for \texttt{Name as
Name}.  A package exposed this way only causes modules (specified before
the \texttt{as}) explicitly listed in the renamings to become visible
under their new names (specified after the \texttt{as}).  For example,
\verb|-package "containers (Data.Set, Data.Map as Map)"| makes
\texttt{Data.Set} and \texttt{Map} (pointing to
\texttt{Data.Map}) available for import.\footnote{See also Cabal files
for a twist on this syntax.}

When the \texttt{-hide-all-packages} flag is applied, uses of the
\texttt{-package} flag are \emph{cumulative}; each argument is processed
and its bindings added to the global module map.  For example,
\verb|-hide-all-packages -package containers -package "containers (Data.Map as Map)"| brings both the default exposed modules of
containers and a binding for \texttt{Map} into scope.\footnote{The
previous behavior, and the current behavior when
\texttt{-hide-all-packages} is not set, is for a second package flag for
the same package name to override the first one.}\footnote{We defer
discussion of what happens when a module name is bound multiple times until
we have discussed signatures, which have interesting behavior on this front.}

\subsection{Disambiguating imports}

With module thinning and renaming, as well as the installed package
database, it is possible for GHC to have multiple bindings for a single
module name.  If the bindings are ambiguous, GHC will report an error
when the user attempts to use the identifier.

Define the \emph{true module} associated with a binding to be the
backing implementation, if the binding is for a signature,\footnote{This
implements signature merging, as otherwise, we would not necessarily
expect original signatures to be equal} and the original module
otherwise.  A binding is unambiguous if the true modules of all the
bindings are equal.  Here is an example of an unambiguous set of exposed
modules:

\begin{verbatim}
exposed-modules:
    A from pkg:AImpl,
    A is pkg:AImpl,
    A from other-pkg:Sig is pkg:AImpl
\end{verbatim}

This mapping says that this package reexports \texttt{pkg:AImpl} as
\texttt{A}, has an \texttt{A.hsig} which was compiled against
\texttt{pkg:AImpl}, and reexports a signature from \texttt{other-pkg}
which itself was compiled against \texttt{pkg:AImpl}.

When Haskell code makes an import, we either load the backing implementation,
if it is available as a direct reexport or original definition, \Red{or else
load \emph{all} of the interface files available as signatures.  Loading
all of the interfaces is guaranteed to not cause conflicts, as the
backing implementation of all the signatures is guaranteed to be identical
(assuming that it is unambiguous.)}

\begin{color}{red}
\paragraph{Home package signatures}  In some circumstances, we may
both define a signature in the home package, as well as import a
signature with the same name from an external package.  While multiple
signatures from external packages are always merged together, in some
cases, we will ignore the external package signature and \emph{only}
use the home package signature: in particular, if an external signature
is not exposed from an explicit \texttt{-package} flag, it is not
merged in.
\end{color}

\paragraph{Package imports} A package import, e.g.,

\begin{verbatim}
import "foobar" Baz
\end{verbatim}

operates as follows: ignore all exposed modules under the name which
were not directly exposed by the package in question.  If the same
package name was included multiple times, all instances of it are
considered (thus, package imports cannot be used to disambiguate
between multiple versions or instantiations of the same package.
For complex disambiguation, use thinning and renaming.)

In particular, package imports consider the \emph{immediate} package
which exposed a module, not the original package which defined the
module.

\paragraph{Typechecking}  \Red{When typechecking only, there is not
necessarily a backing implementation associated with a signature.  In
this case, even if the original names match up, we must perform an
\emph{additional} check to ensure declarations have compatible types.}
This check is not necessary during compilation, because \texttt{-sig-of}
will ensure that the signatures are compatible with a common, unique
backing implementation.

\begin{color}{red}
\paragraph{User-interface}  A number of operations in the compiler
accept a module name, and perform some operation assuming that, if
the name successfully resolves, it will identify a unique module.  In
the presence of signatures, this assumption no longer holds.   In this
section, we describe how to adjust the behavior of these various
operations:

\begin{itemize}
    \item \verb|ghc --abi-hash M| fails if \texttt{M} resolves to multiple
        signatures.  Same rules for home/external package resolution apply,
        so in the absence of any other flags we will hash the signature
        interface in the home package.
    \item
\end{itemize}
\end{color}

\subsection{Indefinite external packages}

\Red{Not implemented yet.}

\section{Backpack}

\Red{This entire section is a proposed and has not been implemented.}

In this section, we describe an expanded version of the package language
described in the Backpack paper which GHC accepts as input.  Given a
\emph{Backpack file}, GHC performs shaping analysis, typechecking,
compilation and registration of multiple packages (whose source code is
specified by the Backpack file).  A Backpack file replaces use of
\texttt{-shape-of}, \texttt{-sig-of} and \texttt{-package} flags.\footnote{Backpack files are \emph{generated} by Cabal.  Cabal is responsible for downloading source files, resolving what versions of packages are to be used, executing conditional statements.  Once the Cabal files are compiled into a Backpack file, it is passed to GHC, which is responsible for instantiating holes and compiling the packages.  The package descriptions in a Backpack file are not full Cabal packages, but contain the minimum information necessary for GHC to work: they are more akin to entries in the installed package database (with some differences).}\footnote{One design goal of this separate package language from Cabal is that it can more easily be incorporated into a language specification, without needing the specification to pull in a full description of Cabal.}

Here is a very simple Backpack file which defines two packages:\footnote{It could have been written as two separate files: the effect of processing this Backpack file is to compile both packages simultaneously.}

\begin{verbatim}
package a
    exposed-modules:     A

package b
    includes:            a
    exposed-modules:     B
\end{verbatim}

Here is a more complicated Backpack file taking advantage of the availability
of signatures in Backpack:

\begin{verbatim}
installed package base
    installed-id:        base-4.0.6-0123456789abcdef

package p
    includes:            base
    exposed-modules:     P
    other-modules:       InternalsP
    required-signatures: Map
    source-dir:          /srv/code/p
    installed-id:        p-2.0.1-abcdef0123456789
                         p-2.0.1-def0123456789abc

package q
    includes:            base, p (Map as QMap)
    exposed-modules:     Q
    other-modules:       QMap
    source-dir:          /srv/code/q
\end{verbatim}

A Backpack file consists of a list of named packages, each of which
is composed of fields (similar to fields in Cabal package description)
which specify various aspects of the package.  A package may optionally
be an \emph{installed} package (specified by the \texttt{installed}
keyword), in which case the package refers to an existing package
(with no holes) in the installed package database; in this case,
all fields are omitted except for \texttt{installed-id}, which identifies the
specific package in use.

All packages in a Backpack file live in the global namespace.
\Red{A possible future addition would be the ability to specify private
packages which are not exposed.}

\begin{verbatim}
backpack ::= package_0
             ...
             package_n

package ::= "package" pkgname
                field_0
                ...
                field_n
          | "installed package" pkgname
                "installed-id:" ipid

pkgname ::= /* package name, e.g. containers (no version!) */

field ::= "includes:"            includes
        | "exposed-modules:"     modnames
        | "other-modules:"       modnames
        | "exposed-signatures:"  modnames
        | "required-signatures:" modnames
        | "reexported-modules:"  reexports
        | "source-dir:"          path
        | pkgdb_field
\end{verbatim}

We now describe the package fields in more detail.

\subsection{\texttt{includes}}

\begin{verbatim}
includes ::= include_0 "," ... "," include_n
include ::= pkgname ["(" renames ")"]

renames ::= rename_0 "," ... "," rename_n
rename ::= modname
         | modname "as" modname
\end{verbatim}

The \texttt{includes} field consists of a comma-separated list of
packages to include.  This field is similar to the Cabal
\texttt{build-depends} field, except that no version numbers are
allowed.  Each package has all exposed modules and signatures are
brought into scope under their original names, unless there is a
parenthesized, comma-separated thinning and renaming specification which
causes only the specified modules are brought into scope (under new
names, if the \texttt{as} keyword is used).  Here is an example
\texttt{includes} field, which brings into scope all exposed modules
from \texttt{base}, \texttt{P1} and \texttt{P2} from \texttt{p}, and
\texttt{Q} from \texttt{q} under the name \texttt{MyQ}.

\begin{verbatim}
    includes: base, p (P1, P2), q (Q as MyQ)
\end{verbatim}

Package inclusion is the mechanism by which holes are instantiated:
a hole and an implementation which are brought in the same scope with
the same name are linked together.  If a package is included multiple
times, it is treated as a separate instantiation for the purpose of
filling holes.

\subsection{\texttt{exposed-modules}, \texttt{other-modules}, \texttt{exposed-signatures}, \texttt{required-signatures}}

\begin{verbatim}
modnames ::= modname_0 ... modname_n
\end{verbatim}

The \texttt{exposed-modules}, \texttt{other-modules},
\texttt{exposed-signatures} and \texttt{required-signatures} are exactly
analogous to their Cabal counterparts, and consist of lists of module names
which are to be compiled from the package's source directory.  For example,
to expose modules \texttt{P} and \texttt{Q}, you would write:

\begin{verbatim}
    exposed-modules: P Q
\end{verbatim}

\subsection{\texttt{reexported-modules}}

\begin{verbatim}
reexports ::= reexport_0 "," ... "," reexport_n
reexport  ::= modname "as" modname
            | modname
\end{verbatim}

The \texttt{reexported-modules} field is exactly analogous to its Cabal
counterpart, and allows reexporting an in-scope module under a different name.\footnote{This is different from \emph{aliasing} in the original Backpack language, since reexported modules are not visible in the current package.}  For example, to reexport a locally available module \texttt{P} under the name \texttt{Q}, write:

\begin{verbatim}
    reexported-modules: P as Q
\end{verbatim}

\subsection{\texttt{source-dir}}

\begin{verbatim}
path ::=  /* file path, e.g. /home/alice/src/containers */
\end{verbatim}

The \texttt{source-dir} field specifies where the source files of
the package in question live, e.g. if \texttt{source-dir: /foo}
then we expect the \texttt{hs} file for module \texttt{A} to live
in \texttt{/foo/A.hs}.

\subsection{\texttt{installed-id}}

\begin{verbatim}
ipid ::= /* installed package ID, e.g. containers-0.8-HASH */
\end{verbatim}

The \texttt{installed-id} field specifies an existing, \emph{compiled} package in
the installed package database, which should be used.  This information
is only used in the case of an \texttt{installed package} entry, because
we would otherwise not have enough information to calculate a package
key for the package.  It is analogous to the \texttt{-package-id} flag.

\Red{This is enough if, in a package database, a given package key is
    unique.  If package keys are not unique, it might also be necessary
    to explicitly provide multiple multiple \texttt{installed-id}s for
an indefinite package, corresponding to valid compilations of the
package with different hole instantiations.  This never happens with
current Cabal, since version numbers are built into package keys.}

\subsection{Installed package database fields}

\begin{verbatim}
    pkgdb_field ::= ...
\end{verbatim}

GHC's installed package database supports a number of other fields
which are necessary for GHC to build some packages, e.g., the \texttt{extraLibraries}
field which specifies operating system libraries which also have to
be linked in.  Backpack packages accept any fields which are valid in the
installed package database, except for: \texttt{name}, \texttt{id}, \texttt{key}
and \texttt{instantiated-with} (which are computed by GHC itself).
The full list of available fields can be found in the \texttt{bin-package-db}
package.

\subsection{Structure of a Backpack file}

In general, a Backpack file must contain the package descriptions of
\emph{all} packages which are transitively depended on (in case
one of those packages must be rebuilt.)  However, if we know a specific
version of a package is already in the installed package database,
its description may be replaced with an \texttt{installed package}
entry, in which case the description (and description of its dependencies)
can be omitted.  \Red{An alternative is to have an indefinite package
database, in which case this database is simply always in scope.  This
might be better if we want to save interface files associated with indefinite
packages.}

It should be emphasized that while the Backpack file leaves the instantiation
of holes implicit (to be resolved by looking at the included packages and
linking modules together), \emph{all package versions} must be resolved
prior to writing a Backpack file.  A Backpack file assumes that the
versions of all packages are consistent (e.g., any reference to \texttt{foo}
will always be a reference to \texttt{foo-1.2}).

% Confusion:
%   - It's not really clear what 'installed package foo' refers to
%   - What does it mean to "install" an indefinite package?
%   - So I guess having the 'installed package' qualifier is not useful,
%     because "indefinite" ones also have precompiled indefinite ones
%   - The Cabal compilation process: write it out
%       1. Cabal copies relevant q-3.4.cabal into .bkp
%       2. Resolves version
%       3. Selects bits GHC needs
%       4. Downloads source code
%       5. Executes conditionals
%   - Want to distinguish different names from installed package
%     database, local names, Hackage names (invariant: Hackage names
%     never show up)
%   - SPJ trap: version resolution versus hole instantiation
%   - Another red herring: couldn't Cabal pick different versions for
%     the same package
%   - Halfway house: definite packages can be snipped off, but
%     put in all the indefinite ones
%       This is BETTER than having an indefinite package database,
%       because all that's doing is saving us from having to write
%       some characters into a file, it doesn't save us compilation
%       time. (So NO INDEFINITE PACKAGE DATABASE)
%   - Update: version versus holes is REALLY CONFUSING (NO HOLES!)
%   - But for TYPECHECKING you probably do want the indefinite package
%     database, for the INTERFACE FILES

\section{Cabal}

\subsection{Fields in the Cabal file}

The Cabal file is a user-facing description of a package, which is
converted into an \texttt{InstalledPackageInfo} during a Cabal build.
Backpack extends the Cabal files with four new fields, all of which
are only valid in the \texttt{library} section of a package:

\paragraph{required-signatures}  A space-separated list of module names
specifying internal signatures (in \texttt{hsig} files) of the package.
\Red{Signatures specified in this field are not put in the \texttt{exposed-modules} field in the installed package database and
are not available for external import}; however, in order for a package to be
compiled, implementations for all of its signatures must be provided (so
they are not completely \emph{hidden} in the same way \texttt{other-modules} are).

\paragraph{exposed-signatures}  A space-separated list of module names
specifying externally visible signatures (in \texttt{hsig} files) of the package.  It is
represented in the installed package database as an \texttt{exposed-module} with a
non-empty backing implementation (\texttt{Sig is Impl}). Signatures exposed in this way are
available for external import.  In order for a package to be compiled,
implementations for all exposed signatures must be provided.

\paragraph{indefinite}  A package is \emph{indefinite} if it has any
uninstantiated
\texttt{required-signatures} or \texttt{exposed-signatures}, or it
depends on an indefinite package without instantiating all of the holes
of that package.  In principle, this parameter can be calculated
by Cabal, but it serves a documentory purpose for packages which do not
have any signatures themselves, but depend on packages which are indefinite.
\Red{Actually, this field is in the top-level at the moment.}

\paragraph{reexported-modules}  A comma-separated list of module or
signature reexports.  It is represented in the installed package
database as a module with a non-empty original module/signature: the
original module is resolved by Cabal.  There are three valid syntactic
forms:

\begin{itemize}
    \item \texttt{Orig}, which reexports any module with the
    name \texttt{Orig} in the current scope (e.g.,
    as specified by \texttt{build-depends}).

    \item \texttt{Orig as New}, which reexports a module with
    the name \texttt{Orig} in the current scope.  \texttt{Orig}
    can be a home module and doesn't necessarily have to come
    from \texttt{build-depends}.

    \item \texttt{package:Orig as New}, which reexports a module
    with name \texttt{Orig} from the specific source package \texttt{package}.
\end{itemize}

If multiple modules with the same name are in scope, we check
if it is unambiguous (the same check used by GHC); if they are
we reexport all of the modules; otherwise, we give an error.
In this way, packages which reexport multiple signatures to the
same name can be valid; a package may also reexport a signature
onto a home \texttt{hsig} signature.

\subsection{build-depends}

This field has been extended with new syntax
to provide the access to GHC's new thinning and renaming functionality
and to have the ability to include an indefinite package \emph{multiple times}
(with different instantiations for its holes).

Here is an example entry in \texttt{build-depends}:
\verb|foo >= 0.8 (ASig as A1, B as B1; ASig as A2, ...)|.  This statement includes the
package \texttt{foo} twice, once with \texttt{ASig} instantiated with
\texttt{A1} and \texttt{B} renamed as \texttt{B1}, and once with
\texttt{ASig} instantiated with \texttt{A2}, and all other modules
imported with their original names.  Assuming that the key of the first
instance of \texttt{foo} is \texttt{foo\_KEY1} and the key of the second instance
is \texttt{foo\_KEY2}, and that \texttt{ASig} is an \texttt{exposed-signature}, then this \texttt{build-depends} would turn into
these flags for GHC\@: \verb|-package-key "foo\_KEY1 (ASig as A1, B as B1)" -package-key "foo\_KEY2" -package-key "foo\_KEY2 (ASig as A2)"|

Syntactically, the thinnings and renamings are placed inside a
parenthetical after the package name and version constraints.
Semicolons distinguish separate inclusions of the package, and the inner
comma-separated lists indicate the thinning/renamings of the module.
You can also write \verb|...|, which simply
includes all of the default bindings from the package.
\Red{This is not implemented. Should this only refer to modules which were not referred to already? Should it refer only to holes?}

There are two remarks that should be made about separate instantiations of
the package.  First, Cabal will automatically ``de-duplicate'' instances of
the package which are equivalent: thus, \verb|foo (A; B)| is equivalent to
\texttt{foo (A, B)} when \texttt{foo} is a definite package, or when the
holes instantiation for each instance is equivalent.  Second, when merging
two \texttt{build-depends} statements together (for example, due to
a conditional section in a Cabal file), they are considered \emph{separate
inclusions of a package.}

\subsection{Setup flags}

There is one new flag for the \texttt{Setup} script, which can be
used to manually provide instantiations for holes in a package:
\verb|--instantiate-with NAME=PKG:MOD|, which binds a module \verb|NAME|
to the implementation \verb|MOD| provided by installed package ID \verb|PKG|.
The flag can be specified multiply times to provide bindings for all
signatures.  The module in question must be the \emph{original} module,
not a re-export.



\subsection{Metadata in the installed package database}

Cabal records

\texttt{instantiated-with}

\section{cabal-install}

\subsection{Indefinite package instantiation}

\end{document}
